\usepackage{relsize}
\renewcommand{\_}{\textscale{.7}{\textunderscore}}
\def\xminimo{1.50}
\def\xmaximo{1.55}
\usepackage{tikz-dimline}
\usepackage{newtxtext}
\usepackage{newtxmath}
\usepackage{scalerel}

\usepgfplotslibrary{units,fillbetween,groupplots,colorbrewer}
\usetikzlibrary{pgfplots.colorbrewer,}
\pgfplotsset{every axis/.style = {
		scale only axis,
		width=8cm,height=5cm,
		line width = 0.7pt,
		tick style = {line width=0.7pt,black},
		ticklabel style={scale=0.8},
		major tick length = 1.5mm,
		minor tick length = 0.75mm,
		xtick pos=left,
		ytick pos=left,
		minor x tick num=1,
		minor y tick num=1,
		xlabel style={scale=0.9},
		ylabel style={scale=0.9},
		zlabel style={scale=0.9},
		xlabel={Photon energy (eV)}, 
	   	xticklabel style={
	   	/pgf/number format/precision=2,
	   	/pgf/number format/fixed,
	   	/pgf/number format/fixed zerofill,
	   },
	   yticklabel style={
	   	/pgf/number format/precision=2,
	   	/pgf/number format/fixed,
	   	/pgf/number format/fixed zerofill,
	   },
	},
	every axis plot/.style={smooth,mark=*,mark options={scale=0.5,fill=white,line width=0.5pt},line width=0.5pt},
	/pgfplots/legend image code/.code={%
		\draw[mark repeat=1,mark phase=1,#1] 
		plot coordinates {
			(0cm,0cm) 
			(0.0cm,0cm)
			(0.0cm,0cm)
			(0.0cm,0cm)
			(0.3cm,0cm)%
		};
	},
}
\pgfplotsset{every axis legend/.style={
		cells={anchor=center},
		inner xsep=1pt,
		inner ysep=1pt,
		nodes={scale=0.6,inner sep=2pt, transform shape},
		draw=none,
		at={(1,1)},
		anchor=north east,
	}
}

% \pgfplotsset{
% 	cycle from colormap manual style/.style={
% 	every axis plot/.style={smooth,mark=*,mark options={scale=0.5,fill=white,line width=0.5pt},line width=0.5pt},
% 	},
% }

% \pgfdeclareplotmark{*)}{\shade[draw=red!70!black,ball color=red!70] (0pt, 0pt) circle [radius=3pt];}
% \pgfdeclareplotmark{**)}{\shade[draw=blue!70!black,ball color=blue!70] (0pt, 0pt) circle [radius=3pt];}
% \pgfdeclareplotmark{***)}{\shade[draw=\rgb{0.31}{0.78}{0.47},ball color=\rgb{0.31}{0.78}{0.1}] (0pt, 0pt) circle
% 	[radius=3pt];}



% Commands 
% As a command
\newcommand{\m}[3]{M4\_3#1#2#3 }
\newcommand{\sm}[1]{M4\_#1}
\DeclareMathSymbol{\mh}{\mathord}{operators}{`\-}
\newcommand\thh[2]{\mathrm{e_{\scaleto{#1\mathstrut}{5pt}} \mh hh_{\scaleto{#2\mathstrut}{5pt}}}}
\newcommand\tlh[2]{\mathrm{e_{\scaleto{#1\mathstrut}{5pt}} \mh lh_{\scaleto{#2\mathstrut}{5pt}}}}
\newcommand\bandt[7]{
\path (axis cs:#1,#2) -- node[midway,scale=#6-0.1,yshift=#5,color=#7,inner sep=1pt] (point) {#4} (axis cs:#1,#3);
	\draw[-{Stealth[scale=#6]},line width=0.7pt,color=#7] (axis cs:#1,#2) -- (point) -- (axis cs:#1,#3);
}


\newcommand\trs[6]{
	\node[scale=0.8,text width = 1em,inner sep = 2pt,transform canvas={xshift=#5}] at (axis cs:#1,#2) {#6} edge [-{Stealth[scale=#4]}] (axis cs:#1,#3);
}



