\newcommand{\Aref}[1]{Appendix  \ref{#1}} %\thechapter
\def\chapterautorefname{Chapter}
\usepackage{xspace}
\definecolor{NatGreen}{RGB}{50,93,61}
\newcolumntype{x}[1]{%
>{\raggedleft\hspace{0pt}}m{#1}}%




\hypersetup
{pdftitle={PhD RUCO Thesis},
pdfauthor={Oscar Ruiz},
pdfsubject={ PhD Thesis}, %subject of the document
%pdftoolbar=false, % toolbar hidden
pdfmenubar=true, %menubar shown
pdfhighlight=/O, %effect of clicking on a link
colorlinks=true, %couleurs sur les liens hypertextes
pdfpagemode=UseOutlines,%UseNone, %aucun mode de page
pdfpagelayout=SinglePage,%SinglePage,TwoPageRight, %ouverture en simple page
pdffitwindow=true, %pages ouvertes entierement dans toute la fenetre
linkcolor=blue, %couleur des liens hypertextes internes
citecolor=blue, %couleur des liens pour les citations
urlcolor=cyan,  %couleur des liens pour les url
bookmarksopenlevel=2
}




%=================================== footnotes options ==============================================================%
\renewcommand{\thefootnote}{\fnsymbol{footnote}}
\newcommand\blfootnote[1]{%
	\begingroup
	\renewcommand\thefootnote{}\footnote{#1}%
	\addtocounter{footnote}{-1}%
	\endgroup
}
%---------------------------------------------------------%
%\DeclareFieldFormat{volcitevolume}{\bibstring{volume}\ppspace\RN{#1}}
%\AtEveryCitekey{\clearfield{institude}}
%\renewcommand{\cite}{\supercite}
%\renewcommand{\citep}{\supercite}
%\renewcommand{\citet}[2][]{\citeauthor{#2}\supercite{#2}\ifstrempty{#1}{}{, #1}}
% \DeclareFieldFormat{bibentrysetcount}{\newline(\mknumalph{#1})\addhighpenspace}


% \renewcommand{\ptcfont}{\normal\rm}
% \renewcommand{\ptcCfont}{\normal\bm}
% \renewcommand{\ptcSfont}{\normal\rm}
% \renewcommand{\ptcSSfont}{\footnotesize\rm}
% \renewcommand{\ptcSSSfont}{\footnotesize\rm}





% Some useful commands and shortcut for maths:  partial derivative and stuff
\newcommand{\pd}[2]{\frac{\partial #1}{\partial #2}}
\def\abs{\operatorname{abs}}
\def\argmax{\operatornamewithlimits{arg\,max}}
\def\argmin{\operatornamewithlimits{arg\,min}}
\def\diag{\operatorname{Diag}}
\newcommand{\eqRef}[1]{(\ref{#1})}



%Paths
%\graphicspath{{../FIGURES/CHAPTER-3}{../FIGURES/CHAPTER-2}}
\graphicspath{{./FIGURES/}}
% My commands
\newcommand{\sch}{Scr\"odinger }
\newcommand{\senergy}{\displaystyle\mathrm{\bf E}}
\newcommand{\rv}{\displaystyle\mathrm{\bf r}}
\newcommand{\GaAs}[1]{\displaystyle\text{#1GaAs}}
\newcommand{\AlAs}[1]{\displaystyle\text{#1AlAs}}
%\newcommand{\bf}[1]{\mathrm{\bfseries #1}}
%\newcommand{\AlGaAs}[1]{\displaystyle\text{#1 Al_{x}Ga_{1-x}As}}
\newcommand{\AlGaAsd}[2]{\displaystyle\text{Al_{#1}Ga_{#2}As}}
\def\algaas{Al$_{x}$Ga$_{1-x}$As }
\def\xp{${\mathbf{X}}^{\mathrm{\ensuremath{+}}}$ }
\def\xm{${\mathbf{X}}^{\mathrm{\ensuremath{-}}}$ }
\def\xhh{${\mathit{X}}_{\mathrm{\ensuremath{hh}}}$ }
\def\xlh{${\mathit{X}}_{\mathrm{\ensuremath{lh}}}$ }


% defining sample names
\def\tusu{M4\_3171 }
\def\tusd{M4\_3171 }
\def\tdvs{M4\_3226 }
\def\tuc{M4\_3140 }   
\def\tucu{M4\_3141 }
\def\tcvu{M4\_3521 }
\def\tcvd{M4\_3522 }
\def\tcvt{M4\_3523 }

\newcommand\bff[1]{\mathrm{\bf #1}}
\newcommand{\bs}[1]{\boldsymbol #1}
% As a command
\newcommand{\m}[3]{M4\_3#1#2#3 }
\newcommand{\sm}[1]{M4\_#1}
\DeclareMathSymbol{\mh}{\mathord}{operators}{`\-}
\newcommand\thh[2]{e_#1 \mh hh_#2}
\newcommand\tlh[2]{e_#1 \mh lh_#2}
% RGB COLOR COMMAND
\newcommand{\rgb}[3]{rgb:red,#1;green,#2;blue,#3}

% defined bold and emph nomenclature
% nomenclature commands and definitions
\def\sym{\emph{symmetry }}
\def\cry{\emph{crystal }}
\def\brill{\emph{Brillouin }}
\def\bz{\textbf{BZ }}
\def\ks{$\boldsymbol{k}$ }
\def\ksm{\boldsymbol{k} }

\newcommand{\bdlo}[2]{$\boldsymbol{#1}(#2)$}
\newcommand{\bdli}[2]{$#1(\boldsymbol{#2})$}
\newcommand{\boldlm}[1]{$\boldsymbol{#1}$}
\newcommand{\boldl}[1]{\boldsymbol{#1}}

