%================================= Coode tools =================================
\setglossarypreamble[abbreviations]
{This list denote the \emph{Open-Source} packages,  codes, tools, and repositories for the development of this
work. All inside of this work as images or numerical calculations are subject to the
\emph{Open-Source} ideology. Our codes are housed in our own GitHub repository, both personal
and laboratory repository. It’s important to say that without the development of the \emph{Open-Source}
codes like contents in this list, our codes, they couldn’t be enhanced.
}

\newglossaryentry{kp-lflm-group}
{ type=abbreviations,
    name={\href{https://github.com/NanophotonIICOs/cqws-codes.git}{cqws-codes}},
    description={Repository of our codes implemented in this work. }
    \cite{cqws-codes}
}
\newglossaryentry{cqws-codes}
{ type=abbreviations,
    name={\href{https://github.com/NanophotonIICOs/kp-nanoiico-group.git}{kp-nanoiico-group}},
    description={\gls{kp} julia\cite{Julia-2017} package developed by Our group research}
    \cite{kp-nanoiico-group}
}
\newglossaryentry{ase}
{ type=abbreviations,
    name={ASE},
    description={The Atomic Simulation Environment (ASE) is a set of tools and Python modules for setting up, 
    manipulating, running, visualizing and analyzing atomistic simulations. \cite{ask2017ase}}
}
\newglossaryentry{Spglib}
{ 
    type=abbreviations,
    name={Spglib},
    description=
    {
    Software library for crystal symmetry search\cite{togo2018textttspglib}
    }
}

\newglossaryentry{solcore}
{ type=abbreviations,
    name={SOLCORE},
    description={A multi-scale, Python-based library for modelling solar cells and semiconductor materials
     \cite{alonso2018solcore}}
}

\newglossaryentry{aestimo}
{ type=abbreviations,
    name={Aestimo},
    description=
    {
    One-dimensional (1D) self-consistent Schrödinger-Poisson solver for 
    semiconductor heterostructures
    \cite{hebal2021general}
    }
}

\newglossaryentry{vesta}
{ type=abbreviations,
    name={VESTA},
    description=
    {
    3D visualization program for structural models, volumetric data such as electron/nuclear 
    densities, and crystal morphologies. 
    \cite{momma2011vesta}
    }
}

\newglossaryentry{pgftikz}
{ 
    type=abbreviations,
    name={PGF/TikZ},
    description=
    {
    PGF is a macro package for creating graphics. It is platform- and format-independent and works together 
    with the most important \TeX backend drivers, including pdf\TeX  and dvips. It comes with a user-friendly 
    syntax layer called TikZ. 
    \cite{pgftikz}
    }
}

\newglossaryentry{optexp}
{ 
    type=abbreviations,
    name={pst-optexp},
    description=
    {
    \href{https://tug.org/PSTricks/main.cgi/}{PStricks} package to drawing optical experimental setups.
    \cite{optexp}
    }
}


%================================ symbols ====================================

\glsxtrnewsymbol
[description = {Negative Trion}]
{ntrion}% label
{\ensuremath{{\mathbf{X}}^{\mathrm{-}}}}% symbol

\glsxtrnewsymbol
[description = {Positive Trion}]
{ptrion}% label
{\ensuremath{{\mathbf{X}}^{\mathrm{\ensuremath{+}}}}}% symbol

\glsxtrnewsymbol
[description = {Direct Exciton}]
{x}% label
{\ensuremath{{\mathbf{X}}}}% symbol


\glsxtrnewsymbol
[description = {Indirect Exciton}]
{ix}% label
{\ensuremath{{\mathbf{IX}}}}% symbol


\glsxtrnewsymbol
[description = {AlGaAs semiconductor as a function of Al concentration $x$}]
{algaas}% label
{\ensuremath{\mathrm{Al}_{x}\mathrm{Ga}_{1-x}\mathrm{As}}}

\glsxtrnewsymbol
[description = {Planck's constant (eV)}]
{hbar}% label
{\ensuremath{\hbar}}


\glsxtrnewsymbol
[description = {electron effective mass}]
{m0}% label
{\ensuremath{m_{0}}}

\glsxtrnewsymbol
[description = {Family of lattice planes with Miller indices $h$, $k$ and $l$}]
{miller}% label
{\ensuremath{(hkl)}}

\glsxtrnewsymbol
[description = {Energy bandgap}]
{eg}% label
{\ensuremath{E_{g}}}


\glsxtrnewsymbol
[description = {eletron}]
{e}% label
{\ensuremath{e}}

\glsxtrnewsymbol
[description = {heavy-hole}]
{hh}% label
{\ensuremath{hh}}

\glsxtrnewsymbol
[description = {light-hole}]
{lh}% label
{\ensuremath{lh}}

\glsxtrnewsymbol
[description = {Electronic transitions}]
{tran}% label
{\ensuremath{\mathrm{e_{n}-hh_{n}}} or \ensuremath{\mathrm{e_{n}-lh_{n}}}}

\glsxtrnewsymbol
[description = {}]
{ }% label
{\ensuremath{}}









%================================= Abbreviations ==============================
% \nomenclature[BIA]{BIA}{Bulk inversion asymmetry}
% \nomenclature[SIA]{SIA}{Structure inversion asymmetry}
% \nomenclature[EMA]{EMA}{Effective Mass Approximation}
% \nomenclature[FWHM]{FWHM}{Full width at half-maximum}
% \nomenclature[m0]{$m_{0}$}{electron effective mass}
% \nomenclature[DX]{\textbf{DX}}{Direct exciton}
% \nomenclature[IX]{\textbf{IX}}{Indirect exciton}
% \nomenclature[Photo-detector]{PD}{Photo-detector}
% \nomenclature[BL]{\textbf{BL}}{Beer-Lambert law}
% \nomenclature[BZ]{\textbf{BZ}}{\emph{Brillouin zone}}
% \nomenclature[2DEG]{2DEG}{Two-dimensional electron gas}
% \nomenclature[TB]{TB}{Tight binding}
% \nomenclature[lattice planes]{$(hkl)$}{Family of lattice planes with Miller indices $h$, $k$ and $l$}

\newglossaryentry{BS}
{type={nomenclature},
name={BS},
description={Band structure},
}

\newglossaryentry{BZ}
{type={nomenclature},
name={BZ},
description={\emph{Brillouin zone}},
}

\newglossaryentry{QS}
{type={nomenclature},
name={QS},
description={Quantum Structures},
}

\newglossaryentry{QW}
{type={nomenclature},
name={QW},
description={Quantum Well},
}

\newglossaryentry{SQW}
{type={nomenclature},
name={SQW},
description={Single Quantum Well},
}

\newglossaryentry{CQWs}
{type={nomenclature},
name={CQWs},
description={Coupled Quantum Wells},
}

\newglossaryentry{VB}
{type={nomenclature},
name={VB},
description={Valence Band},
}


\newglossaryentry{CB}
{type={nomenclature},
name={CB},
description={Conduction Band},
}

\newglossaryentry{SCQWs}
{type={nomenclature},
name={SCQWs},
description={Symmetric coupled quantum wells},
}

\newglossaryentry{ACQWs}
{type={nomenclature},
name={ACQWs},
description={Asymmetric coupled quantum wells},
}

\newglossaryentry{RAS}
{type={nomenclature},
name={RAS},
description={Reflectance Anisotropy Spectroscopy},
}

\newglossaryentry{PL}
{type={nomenclature},
name={PL},
description={Photoluminiscense spectroscopy},
}
\newglossaryentry{PR}
{type={nomenclature},
name={PR},
description={Photoreflectance spectroscopy},
}
\newglossaryentry{R}
{type={nomenclature},
name={R},
description={Reflectance spectroscopy},
}
\newglossaryentry{PRD}
{type={nomenclature},
name={PRD},
description={Photo-Reflectance Differential Spectroscopy },
}
\newglossaryentry{FDM}
{type={nomenclature},
name={FDM},
description={Finite differnce method},
}

\newglossaryentry{ccd}
{type={nomenclature},
name={CCD},
description={Charge coupled device},
}

\newglossaryentry{0d}
{type={nomenclature},
name={0D},
description={Zero-dimensional},
}

\newglossaryentry{1d}
{type={nomenclature},
name={1D},
description={One-dimensional},
}

\newglossaryentry{2d}
{type={nomenclature},
name={2D},
description={Two-dimensional},
}

\newglossaryentry{3d}
{type={nomenclature},
name={3D},
description={Three-dimensional},
}

\newglossaryentry{fcc}
{type={nomenclature},
name={fcc},
description={Face-centered cubic},
}


\newglossaryentry{2dgas}
{type={nomenclature},
name={2DEG},
description={Two-dimensional electron gas},
}

\newglossaryentry{bl}
{type={nomenclature},
name={BL},
description={Beer-Lambert-Law},
}

\newglossaryentry{tb}
{type={nomenclature},
name={TB},
description={Tight-Binding method},
}

\newglossaryentry{pd}
{type={nomenclature},
name={PD},
description={Photo-Detector},
}

\newglossaryentry{pem}
{type={nomenclature},
name={PEM},
description={Photo-Elastic Modulator},
}

\newglossaryentry{qm}
{type={nomenclature},
name={QM},
description={Quantum Mechanics},
}

\newglossaryentry{kp}
{type={nomenclature},
name={$\boldsymbol{k}\bigcdot \boldsymbol{p}$},
description={Semiempirical theoretical tool to calculate band-structure},
}


\newglossaryentry{TB}
{type={nomenclature},
name={TB},
description={Semiempirical Thight-Binding Method},
}

\newglossaryentry{DFT}
{type={nomenclature},
name={DFT},
description={Density Functional Theory},
}

\newglossaryentry{soc}
{type={nomenclature},
name={SOC},
description={Spin-Orbit Coupling, also called Spin-Orbit interaction},
}

\newglossaryentry{nanoiico}
{type={nomenclature},
name={NanophotonIICOs},
description={\href{https://github.com/NanophotonIICOs}{Nanophotonics IICO group}.},
}

\newglossaryentry{efa}
{type={nomenclature},
name={EFA},
description={Envelope function Approximation},
}

\newglossaryentry{ema}
{type={nomenclature},
name={EMA},
description={Effective Mass Approximation},
}

\newglossaryentry{fkos}
{type={nomenclature},
name={FKOs},
description={Franz Keldysh oscillations},
}

\newglossaryentry{oa}
{type={nomenclature},
name={IOA},
description={In-plane Optical Anisotropy},
}


% \newglossaryentry{hj}
% {type={nomenclature},
% name={HJ},
% description={Heterojuntion},
% }


\newglossaryentry{}
{type={nomenclature},
name={},
description={},
}

\renewcommand{\glsnamefont}[1]{\textcolor{black}{\textbf{#1}}}
%\renewcommand*{\glstextformat}[1]{\textcolor{green}{#1}}
\printunsrtglossary[type=nomenclature,style=long3col]
\printunsrtglossary[type=abbreviations,title={List of codes and packages}]
\printunsrtglossary[type=symbols,]

