\cleardoublepage
\vspace{-5cm}
\begin{vcentrepage}
	\noindent\rule[2pt]{\textwidth}{0.8pt}\\
	\begin{center}
		{\Large\textbf{}}
	\end{center}
	{\large\textbf{Abstract}}\\
	In the present work, we were able to identify and characterize a new source of IOAs occurring in asymmetric coupled quantum wells ACQWs; namely a reduction of the symmetry from $D_{2d}$ to $C_{2v}$ as imposed by asymmetry along the growth direction.
	We report  on reflectance anisotropy spectroscopy (RAS) of  double GaAs quantum wells (DQWs)  structures coupled by a thin ($<2$ nm) tunneling  barrier. Two groups of DQWs systems were studied: one where both QWs have the same thickness (symmetric CQWs) and another one where they have different thicknesses (asymmetric DQW). RAS measures the in-plane optical anisotropies (IOAs) arising from the intermixing of the heavy- and light- holes in the valence band when the symmetry of the DQW system is lowered from $D_{2d}$ to $C_{2v}$. If the CQWS are symmetric, residual IOAs stem from the asymmetry of the QW  interfaces; for instance, associated to Ga segregation into the AlGaAs layer during the epitaxial growth process. In the case of an asymmetric CWQs with QWs with different thicknesses, the AlGaAs layers (that are sources of anisotropies) are not distributed symmetrically at both sides of the tunneling barrier. Thus, the system losses its inversion symmetry  yielding an increase of the RAS strength. The RAS line shapes were compared with reflectance spectra in order to assess the heavy- and light- hole mixing induced by the  symmetry breakdown. The energies of the optical transitions were calculated by numerically solving the one-dimensional Schrödinger equation using a finite-differences method. Our results are useful for interpretation of the transitions occurring in both, symmetric and asymmetric CQWs.
	
	\noindent\rule[2pt]{\textwidth}{0.8pt}
\end{vcentrepage}


%