
\section{Introducción}
\subsection{Estructuras de Pozos Acoplados}
\begin{frame}[t]
\frametitle{Introduccion}
\framesubtitle{}
\vspace{-0.5cm}
\begin{tikzpicture}[remember picture, overlay]
\node<1->[anchor = north west, text width = 5cm, yshift = -0.75cm](txt) at (current page.north west) {
	\begin{tcolorbox}[enhanced,title = Experimentos Realizados,
	fonttitle=\bfseries,
	left=1mm,
	top=1mm,
	bottom=1mm,
	right=1mm,
	width = 5cm,
	height=2.25cm,
	boxsep = 0cm,
	coltitle=green!25!black,
	attach boxed title to top center={yshift=-2mm,yshifttext=-1mm},
	boxed title style={colframe=green!75!black,
		colback=yellow!50!green}]
	\begin{itemize}
	\fontsize{8}{1}\selectfont
	\item<1-> Estructuras de Pozos cu\'anticos acoplados
	%\item<1-> Se realizaron calculos de las energ\'ias y las funciones de onda.

	\end{itemize}
	\end{tcolorbox}	
};








\node<1-5>[anchor= north east,xshift=-2cm,yshift=-0.2 5cm] at (current page.north east){\includegraphics[scale=0.25]{Figures/STRUCTURES/FIG1}};

\node<2-5>[xshift = -1cm,yshift=-3.25cm] at (current page.center){
	\begin{tabular}{cccc} \toprule
	Sample number                     & $d_{1} (nm)$ & $d_{2} (nm)$ & $\mathrm{Al_{0.15}Ga_{0.85}As}$ \\ \midrule
	\myrowcolour
	$\mathrm{M4\_3140}$               & 13.85        & 11.87        & Si doped, $\mathrm{n=6\times 10^{18} cm^{-3}}$\\ 	
	\visible<3-5>{$\mathrm{M4\_3523}$ & 11.87        & 11.87        & Si doped, $\mathrm{n=6\times 10^{18} cm^{-3}}$}\\ 	
	\visible<4-5>{
		{$\mathrm{M4\_3521}$}         & 23.74        & 11.87        & Si doped, $\mathrm{n=6\times 10^{18} cm^{-3}}$}\\ 
	\visible<5>{$\mathrm{M4\_3522}$   & 23.74        & 11.87        & Be doped, $\mathrm{p=5\times 10^{19} cm^{-3}}$}\\ \bottomrule
	\end{tabular}
};



\node<6->[anchor=north,
           color=blue,
           text width=5cm,
           font=\fontsize{8pt}{1}\selectfont,
           xshift=0.5cm] at (txt.south){
	$\left[- \dfrac{\hbar^{2}}{2}\partial_{z} \dfrac{1}{m^{*}(z)} \partial_{z} + V(z) - E  \psi(z)\right]=0$}; 




\end{tikzpicture}
\end{frame}


