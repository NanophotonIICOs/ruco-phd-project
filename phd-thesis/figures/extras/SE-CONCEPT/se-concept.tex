\documentclass{article}
\usepackage{tikz,pgfplots}
\usetikzlibrary{math,calc, chains, positioning}
\usepackage[paperheight=5cm,paperwidth=5cm]{geometry}
\usepackage{fontspec}
\setmainfont{xkcd-Regular}
%\setmathrm{Humor-Sans.ttf}[Extension = .ttf ]
%\setmathsf{Humor-Sans.ttf}[Extension = .ttf ]




\begin{document}
	\thispagestyle{empty}
	\begin{tikzpicture}[remember picture,overlay,decoration={random steps,segment length=2mm,amplitude=0.2pt}]
		% You can see the border of the page node with this:
		%\draw[thin] (current page.south west) rectangle (current page.north east);
		
		\node[anchor=west,yshift=0.2cm,scale=0.7](p1) at (current page.west) {$\Psi\mathrm{(hv)}$};
		\node[anchor=north,scale=0.7](p2) at (p1.south) {$\Delta\mathrm{(hv)}$};
		
		\draw[-stealth,line width=2pt,decorate] ([xshift=0.75cm]current page.west) to[out=0,in=180] (current page.center);
		
		\node[text width=2cm, align=center,scale=0.5,anchor=center,yshift=-5cm,xshift=-0.5cm] (current page.center){Construction of optical models};
		
		
		
		
		\node[anchor=north east,
			draw,
			text width=4cm, 
			scale=0.5,inner sep=1mm](ops) at (current page.north east){ Optical constants
		\begin{itemize}
			\item $\mathrm{n,k}$
			\item $\mathrm{\epsilon = \epsilon_{1}-i\epsilon_{3}}$
			\item $\mathrm{\alpha=4\pi k/\lambda}$
		\end{itemize}	
		};
	
	\node[anchor=east,
	draw,
	text width=4cm,minimum height=2cm,minimum width=3cm, 
	scale=0.5,](ops2) at (current page.east){ \begin{itemize}\vspace*{-5mm}
			\item Reflectance
			\item Transmitance
		\end{itemize}	
	};



\node[anchor=south east,
draw,
text width=4cm,minimum height=2cm,minimum width=3cm, 
scale=0.5,](ops3) at (current page.south east){ \begin{itemize}
		\item Surface roughness layer
		\item Bulk layer
		\item Multilayer
	\end{itemize}	
};
		
\draw[-stealth,line width=2pt,decorate] ([xshift=0.75cm]current page.west) to[out=0,in=180,distance=2cm] (ops3.west);	
\draw[-stealth,line width=2pt,decorate] ([xshift=0.75cm]current page.west) to[out=0,in=180,distance=2cm] (ops.west);			
	\end{tikzpicture}
\end{document}