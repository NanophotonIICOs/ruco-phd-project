% \documentclass{article}
% \usepackage[paperwidth = 10cm, 
%             paperheight =5.65cm, 
%             textwidth = 10cm,
%             textheight = 5.65cm,
%             centering]{geometry}
\documentclass[margin=0mm]{standalone}

\usepackage[utf8]{inputenc}
\usepackage{lmodern}
\usepackage{graphics}
\usepackage{tikz,filecontents, pgfplots}
\pgfplotsset{compat=1.3}
\usetikzlibrary{calc,arrows,arrows.meta,shapes,shadows,shapes.arrows,spy,angles,animations,backgrounds,decorations,patterns,babel,bending}
\usepackage{pgfplotstable}
\usepackage{siunitx}
\usepackage{gensymb}
\usepackage{amsmath}
\usepackage{relsize}
\renewcommand{\_}{\textscale{.7}{\textunderscore}}
\def\xminimo{1.50}
\def\xmaximo{1.55}
\usepackage{tikz-dimline}
\usepackage{newtxtext}
\usepackage{newtxmath}
\usepackage{scalerel}

\usepgfplotslibrary{units,fillbetween,groupplots,colorbrewer,colormaps}
\usetikzlibrary{pgfplots.colorbrewer}


\newcommand{\m}[3]{M4\_3#1#2#3 }
\newcommand{\sm}[1]{M4\_#1}
\DeclareMathSymbol{\mh}{\mathord}{operators}{`\-}
\newcommand\thh[2]{\mathrm{e_{\scaleto{#1\mathstrut}{5pt}} \mh hh_{\scaleto{#2\mathstrut}{5pt}}}}
\newcommand\tlh[2]{\mathrm{e_{\scaleto{#1\mathstrut}{5pt}} \mh lh_{\scaleto{#2\mathstrut}{5pt}}}}
\newcommand\bandt[7]{
\path (axis cs:#1,#2) -- node[midway,scale=#6-0.1,yshift=#5,color=#7,inner sep=1pt] (point) {#4} (axis cs:#1,#3);
	\draw[-{Stealth[scale=#6]},line width=0.7pt,color=#7] (axis cs:#1,#2) -- (point) -- (axis cs:#1,#3);
}


\newcommand\trs[7]{
	\node[scale=#7,text width = 1em,inner sep = 2pt,transform canvas={xshift=#5}] at (axis cs:#1,#2) {#6} edge [-{Stealth[scale=#4]}] (axis cs:#1,#3);
}

\pgfplotsset{
height=15cm,
width= 12cm,
every axis/.append style = {
		line width = 1.2pt,
		tick style = {line width=1.2pt,black},
		every tick label/.append style={scale=1.5},
		major tick length = 3mm,
		minor tick length = 1.5mm,
		%axis line style = thick,
	},
every axis plot/.style={smooth,mark=*,mark options={scale=1,fill=white,line width=0.5pt},line width=0.5pt},
x tick label style={
	/pgf/number format/.cd,
	fixed,
	fixed zerofill,
	precision=2,
	/tikz/.cd
},
ytick=\empty,
xmin=1.50,  xmax=1.55,
%enlargelimits = false,
minor x tick num=1,
minor y tick num=1,
xtick pos=left,
ytick pos=left,
xtick distance=1e-2,
xlabel ={ Photon Energy (eV) },
ylabel ={RAS},
every axis x label/.style= {at={(axis description cs:0.5,-0.08)},scale=1.8},
every axis y label/.style= {at={(axis description cs:-0.05,0.5)},rotate=90,scale=1.8}
}


\pgfplotsset{every axis legend/.style={
		cells={anchor=center},
		inner xsep=1pt,
		inner ysep=1pt,
		nodes={scale=1.2,inner sep=2pt, transform shape},
		draw=none,
		at={(1,0.15)},
		anchor=north east,
        legend cell align={left},
	},
}

\pgfplotstableread[col sep=comma]{../../../../scripts/data/ras-m43140-paper1-fix.dat}\rasuno
\pgfplotstableread[col sep=comma]{../../../../scripts/data/RAS-M4_3521-PAPER1.dat}\rastres
\pgfplotstableread[col sep=comma]{../../../../scripts/data/RAS-M4_3523-PAPER1.dat}\rascinco


\begin{document}
\begin{tikzpicture}
\begin{axis}[
	minor x tick num=1,
	ytick=\empty,
	xmin = \xminimo,xmax=\xmaximo,
    ymin=-12e-4, ymax=24e-4,
	]
	
	
    %SCQWS-1
    \addplot[color=Set1-C] table[x index=0,y expr={\thisrowno{3}+18e-4}]\rascinco;
	 \addlegendentry{SCQWs};
	\draw (axis cs:0,0)--(axis cs:1.5025,0);
	\coordinate (cup) at (axis cs:1.51,22e-4);

    % TRANSIITIONS	
    \trs{1.5328}{16e-4}{21e-4}{1}{-4mm}{$\thh{1}{1}$}{1.2}
    \trs{1.5380}{23e-4}{18e-4}{1}{0mm}{$\tlh{1}{1}$}{1.2}
	

    %ACQWS-1
	\addplot[color=Set1-B] table[x index=0,y expr=\thisrowno{4} + 13e-4]\rasuno;
	\addlegendentry{ACQWS-1};
	\draw (axis cs:0,0)--(axis cs:1.5025,0);


    %TRANSITIONS
    \trs{1.5265}{7e-4}{12e-4}{1}{-3mm}{$\thh{1}{1}$}{1.2}
    \trs{1.5296}{14e-4}{9e-4}{1}{0mm}{$\tlh{1}{1}$}{1.2}
{1.2}
    %ACQWS-2
	\addplot[color=Set1-A] table[x index=0,y expr=\thisrowno{5}]\rastres;
	\addlegendentry{ACQWS-2};
	\draw (axis cs:0,0)--(axis cs:1.5025,0);

    %TRANSITIONS
    \trs{1.5181}{-10e-4}{-5e-4}{1}{-6mm}{$\thh{1}{1}$}{1.2}
    \trs{1.5206}{5e-4}{0e-4}{1}{0mm}{$\tlh{1}{1}$}{1.2}
    \trs{1.5340}{-5e-4}{0e-4}{1}{-2mm}{$\thh{2}{2}$}{1.2}
    \trs{1.5394}{5e-4}{0e-4}{1}{-2mm}{$\tlh{2}{2}$}{1.2}

	% \node[scale=0.8,text width = 1em,inner sep = 2pt,transform canvas={xshift=-6.5mm}] (S1H1) at (axis cs:1.5181,-5e-4) {$\thh{1}{1}$} edge [-{Stealth[scale=0.8]}] (axis cs:1.5181,0e-4);

	% \bandt{1.5326}{11e-4}{2e-4}{$\thh{2}{2}$}{2}{\sc}{Set1-C}	
	% \bandt{1.5396}{-13e-4}{-3e-4}{$\tlh{2}{2}$}{-3}{\sc}{Set1-C}	

	
    \dimline[label style = {rotate=-90,midway,xshift=1mm,scale=1},
	line style = {line width=0.5pt,arrows={Stealth[scale=1]}-{Stealth[scale=1]}},
	extension end length=0.3,
	extension start length=0.3,
	] {(axis cs:1.5127,1e-4)}{(axis cs: 1.5127,8e-4)}{$\mathrm{7\!\times\!10^{-4}}$};

    \bandt{1.51}{-12e-4}{-7.3e-4}{Substrate}{2}{1.2}{black};


	\coordinate (cud) at (axis cs:1.51,-7e-4);


\end{axis}

\draw[dashed,line width=0.5pt] (cud)--(cup);


\end{tikzpicture}
\end{document}